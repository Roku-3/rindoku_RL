% Options for packages loaded elsewhere
\PassOptionsToPackage{unicode}{hyperref}
\PassOptionsToPackage{hyphens}{url}
\PassOptionsToPackage{dvipsnames,svgnames,x11names}{xcolor}
%
\documentclass[
  letterpaper,
  DIV=11,
  numbers=noendperiod]{scrreprt}

\usepackage{amsmath,amssymb}
\usepackage{lmodern}
\usepackage{iftex}
\ifPDFTeX
  \usepackage[T1]{fontenc}
  \usepackage[utf8]{inputenc}
  \usepackage{textcomp} % provide euro and other symbols
\else % if luatex or xetex
  \usepackage{unicode-math}
  \defaultfontfeatures{Scale=MatchLowercase}
  \defaultfontfeatures[\rmfamily]{Ligatures=TeX,Scale=1}
\fi
% Use upquote if available, for straight quotes in verbatim environments
\IfFileExists{upquote.sty}{\usepackage{upquote}}{}
\IfFileExists{microtype.sty}{% use microtype if available
  \usepackage[]{microtype}
  \UseMicrotypeSet[protrusion]{basicmath} % disable protrusion for tt fonts
}{}
\makeatletter
\@ifundefined{KOMAClassName}{% if non-KOMA class
  \IfFileExists{parskip.sty}{%
    \usepackage{parskip}
  }{% else
    \setlength{\parindent}{0pt}
    \setlength{\parskip}{6pt plus 2pt minus 1pt}}
}{% if KOMA class
  \KOMAoptions{parskip=half}}
\makeatother
\usepackage{xcolor}
\setlength{\emergencystretch}{3em} % prevent overfull lines
\setcounter{secnumdepth}{5}
% Make \paragraph and \subparagraph free-standing
\ifx\paragraph\undefined\else
  \let\oldparagraph\paragraph
  \renewcommand{\paragraph}[1]{\oldparagraph{#1}\mbox{}}
\fi
\ifx\subparagraph\undefined\else
  \let\oldsubparagraph\subparagraph
  \renewcommand{\subparagraph}[1]{\oldsubparagraph{#1}\mbox{}}
\fi

\usepackage{color}
\usepackage{fancyvrb}
\newcommand{\VerbBar}{|}
\newcommand{\VERB}{\Verb[commandchars=\\\{\}]}
\DefineVerbatimEnvironment{Highlighting}{Verbatim}{commandchars=\\\{\}}
% Add ',fontsize=\small' for more characters per line
\usepackage{framed}
\definecolor{shadecolor}{RGB}{241,243,245}
\newenvironment{Shaded}{\begin{snugshade}}{\end{snugshade}}
\newcommand{\AlertTok}[1]{\textcolor[rgb]{0.68,0.00,0.00}{#1}}
\newcommand{\AnnotationTok}[1]{\textcolor[rgb]{0.37,0.37,0.37}{#1}}
\newcommand{\AttributeTok}[1]{\textcolor[rgb]{0.40,0.45,0.13}{#1}}
\newcommand{\BaseNTok}[1]{\textcolor[rgb]{0.68,0.00,0.00}{#1}}
\newcommand{\BuiltInTok}[1]{\textcolor[rgb]{0.00,0.23,0.31}{#1}}
\newcommand{\CharTok}[1]{\textcolor[rgb]{0.13,0.47,0.30}{#1}}
\newcommand{\CommentTok}[1]{\textcolor[rgb]{0.37,0.37,0.37}{#1}}
\newcommand{\CommentVarTok}[1]{\textcolor[rgb]{0.37,0.37,0.37}{\textit{#1}}}
\newcommand{\ConstantTok}[1]{\textcolor[rgb]{0.56,0.35,0.01}{#1}}
\newcommand{\ControlFlowTok}[1]{\textcolor[rgb]{0.00,0.23,0.31}{#1}}
\newcommand{\DataTypeTok}[1]{\textcolor[rgb]{0.68,0.00,0.00}{#1}}
\newcommand{\DecValTok}[1]{\textcolor[rgb]{0.68,0.00,0.00}{#1}}
\newcommand{\DocumentationTok}[1]{\textcolor[rgb]{0.37,0.37,0.37}{\textit{#1}}}
\newcommand{\ErrorTok}[1]{\textcolor[rgb]{0.68,0.00,0.00}{#1}}
\newcommand{\ExtensionTok}[1]{\textcolor[rgb]{0.00,0.23,0.31}{#1}}
\newcommand{\FloatTok}[1]{\textcolor[rgb]{0.68,0.00,0.00}{#1}}
\newcommand{\FunctionTok}[1]{\textcolor[rgb]{0.28,0.35,0.67}{#1}}
\newcommand{\ImportTok}[1]{\textcolor[rgb]{0.00,0.46,0.62}{#1}}
\newcommand{\InformationTok}[1]{\textcolor[rgb]{0.37,0.37,0.37}{#1}}
\newcommand{\KeywordTok}[1]{\textcolor[rgb]{0.00,0.23,0.31}{#1}}
\newcommand{\NormalTok}[1]{\textcolor[rgb]{0.00,0.23,0.31}{#1}}
\newcommand{\OperatorTok}[1]{\textcolor[rgb]{0.37,0.37,0.37}{#1}}
\newcommand{\OtherTok}[1]{\textcolor[rgb]{0.00,0.23,0.31}{#1}}
\newcommand{\PreprocessorTok}[1]{\textcolor[rgb]{0.68,0.00,0.00}{#1}}
\newcommand{\RegionMarkerTok}[1]{\textcolor[rgb]{0.00,0.23,0.31}{#1}}
\newcommand{\SpecialCharTok}[1]{\textcolor[rgb]{0.37,0.37,0.37}{#1}}
\newcommand{\SpecialStringTok}[1]{\textcolor[rgb]{0.13,0.47,0.30}{#1}}
\newcommand{\StringTok}[1]{\textcolor[rgb]{0.13,0.47,0.30}{#1}}
\newcommand{\VariableTok}[1]{\textcolor[rgb]{0.07,0.07,0.07}{#1}}
\newcommand{\VerbatimStringTok}[1]{\textcolor[rgb]{0.13,0.47,0.30}{#1}}
\newcommand{\WarningTok}[1]{\textcolor[rgb]{0.37,0.37,0.37}{\textit{#1}}}

\providecommand{\tightlist}{%
  \setlength{\itemsep}{0pt}\setlength{\parskip}{0pt}}\usepackage{longtable,booktabs,array}
\usepackage{calc} % for calculating minipage widths
% Correct order of tables after \paragraph or \subparagraph
\usepackage{etoolbox}
\makeatletter
\patchcmd\longtable{\par}{\if@noskipsec\mbox{}\fi\par}{}{}
\makeatother
% Allow footnotes in longtable head/foot
\IfFileExists{footnotehyper.sty}{\usepackage{footnotehyper}}{\usepackage{footnote}}
\makesavenoteenv{longtable}
\usepackage{graphicx}
\makeatletter
\def\maxwidth{\ifdim\Gin@nat@width>\linewidth\linewidth\else\Gin@nat@width\fi}
\def\maxheight{\ifdim\Gin@nat@height>\textheight\textheight\else\Gin@nat@height\fi}
\makeatother
% Scale images if necessary, so that they will not overflow the page
% margins by default, and it is still possible to overwrite the defaults
% using explicit options in \includegraphics[width, height, ...]{}
\setkeys{Gin}{width=\maxwidth,height=\maxheight,keepaspectratio}
% Set default figure placement to htbp
\makeatletter
\def\fps@figure{htbp}
\makeatother

\KOMAoption{captions}{tableheading}
\makeatletter
\makeatother
\makeatletter
\@ifpackageloaded{bookmark}{}{\usepackage{bookmark}}
\makeatother
\makeatletter
\@ifpackageloaded{caption}{}{\usepackage{caption}}
\AtBeginDocument{%
\ifdefined\contentsname
  \renewcommand*\contentsname{Table of contents}
\else
  \newcommand\contentsname{Table of contents}
\fi
\ifdefined\listfigurename
  \renewcommand*\listfigurename{List of Figures}
\else
  \newcommand\listfigurename{List of Figures}
\fi
\ifdefined\listtablename
  \renewcommand*\listtablename{List of Tables}
\else
  \newcommand\listtablename{List of Tables}
\fi
\ifdefined\figurename
  \renewcommand*\figurename{Figure}
\else
  \newcommand\figurename{Figure}
\fi
\ifdefined\tablename
  \renewcommand*\tablename{Table}
\else
  \newcommand\tablename{Table}
\fi
}
\@ifpackageloaded{float}{}{\usepackage{float}}
\floatstyle{ruled}
\@ifundefined{c@chapter}{\newfloat{codelisting}{h}{lop}}{\newfloat{codelisting}{h}{lop}[chapter]}
\floatname{codelisting}{Listing}
\newcommand*\listoflistings{\listof{codelisting}{List of Listings}}
\makeatother
\makeatletter
\@ifpackageloaded{caption}{}{\usepackage{caption}}
\@ifpackageloaded{subcaption}{}{\usepackage{subcaption}}
\makeatother
\makeatletter
\@ifpackageloaded{tcolorbox}{}{\usepackage[many]{tcolorbox}}
\makeatother
\makeatletter
\@ifundefined{shadecolor}{\definecolor{shadecolor}{rgb}{.97, .97, .97}}
\makeatother
\makeatletter
\makeatother
\ifLuaTeX
  \usepackage{selnolig}  % disable illegal ligatures
\fi
\IfFileExists{bookmark.sty}{\usepackage{bookmark}}{\usepackage{hyperref}}
\IfFileExists{xurl.sty}{\usepackage{xurl}}{} % add URL line breaks if available
\urlstyle{same} % disable monospaced font for URLs
\hypersetup{
  pdftitle={強化学習 輪読会},
  colorlinks=true,
  linkcolor={blue},
  filecolor={Maroon},
  citecolor={Blue},
  urlcolor={Blue},
  pdfcreator={LaTeX via pandoc}}

\title{強化学習 輪読会}
\author{}
\date{}

\begin{document}
\maketitle
\ifdefined\Shaded\renewenvironment{Shaded}{\begin{tcolorbox}[boxrule=0pt, sharp corners, interior hidden, enhanced, borderline west={3pt}{0pt}{shadecolor}, frame hidden, breakable]}{\end{tcolorbox}}\fi

\renewcommand*\contentsname{Table of contents}
{
\hypersetup{linkcolor=}
\setcounter{tocdepth}{2}
\tableofcontents
}
\bookmarksetup{startatroot}

\hypertarget{ux8f2aux8aadux4f1aux306bux3042ux305fux3063ux3066}{%
\chapter{輪読会にあたって}\label{ux8f2aux8aadux4f1aux306bux3042ux305fux3063ux3066}}

参加者: ろく、峻平

第1章以降、とくに第Ⅱ部(第9章〜)からはそれぞれが独立した内容が多い。

最初にすべての難易度を把握するのは難しいので、毎週次回の担当範囲を決めることとする。

本の内容に沿って解説していき、わからないところがあれば聞き手はよこやりを入れて質問をしてよい。

担当者はすべての質問に答えられるように準備する。

\hypertarget{ux5f53ux66f8ux3067ux4f7fux308fux308cux308bux8a18ux53f7}{%
\section{当書で使われる記号}\label{ux5f53ux66f8ux3067ux4f7fux308fux308cux308bux8a18ux53f7}}

\begin{Shaded}
\begin{Highlighting}[]
\DecValTok{1} \OperatorTok{+} \DecValTok{1}
\end{Highlighting}
\end{Shaded}

\begin{verbatim}
2
\end{verbatim}

\hypertarget{ux4ecaux5f8cux306eux9032ux3081ux65b9}{%
\section{今後の進め方}\label{ux4ecaux5f8cux306eux9032ux3081ux65b9}}

二人で共通の文書を編集する形式に賛成であれば、次回までにこの文書をサーバーにアップロードし、峻平への導入方法を考えておく。

\begin{longtable}[]{@{}cll@{}}
\toprule()
日時 & ろく & 峻平 \\
\midrule()
\endhead
11/30 & 序章 & \\
12/5 & 第一章 & \\
12/12 & 第二章前半? & 第二章後半? \\
12/19 & & \\
12/26 & & \\
1/2 & & \\
\bottomrule()
\end{longtable}

そうでなければ、例えば一人はbook形式、もう一人はスライドを用いて発表する。

\hypertarget{ux6587ux66f8ux306eux7de8ux96c6ux65b9ux6cd5}{%
\section{文書の編集方法}\label{ux6587ux66f8ux306eux7de8ux96c6ux65b9ux6cd5}}

\begin{enumerate}
\def\labelenumi{\arabic{enumi}.}
\tightlist
\item
  git環境を整える。(wslがおすすめ)
\item
  \href{https://posit.co/download/rstudio-desktop/}{rstudio}と\href{https://quarto.org/docs/get-started/}{quarto}をインストール
\item
  \url{https://github.com/Roku-3/rindoku_RL}にアクセスし、mainブランチでリポジトリをクローンする。rstudioで開く。
\end{enumerate}

\hypertarget{ux7de8ux96c6ux5185ux5bb9ux3092ux9069ux5fdcux3059ux308b}{%
\subsection{編集内容を適応する}\label{ux7de8ux96c6ux5185ux5bb9ux3092ux9069ux5fdcux3059ux308b}}

gitの基本コマンドは以下の通り。

\begin{Shaded}
\begin{Highlighting}[]
\FunctionTok{git}\NormalTok{ pull origin HEAD            }\CommentTok{\# githubからローカルに差分をもってくる}

\FunctionTok{git}\NormalTok{ add .                       }\CommentTok{\# 全てのファイルをコミット対象にする}

\FunctionTok{git}\NormalTok{ commit }\AttributeTok{{-}m} \StringTok{"edit chapter 2"}  \CommentTok{\# 編集内容を表すコメント付きでコミットする。}
                                \CommentTok{\# {-}mはコミットメッセージを一行にするという意味}
                                
\FunctionTok{git}\NormalTok{ push origin HEAD            }\CommentTok{\# githubに情報を送る}
\end{Highlighting}
\end{Shaded}

2人が同じ場所を変更していた場合、コンフリクトが発生する。その時は必ず解消してからpushを行う。

\texttt{git\ branch}と入力したときに\texttt{main}になっていることを確認してからpushすること。\\
pushが正常に行われると自動でデプロイ(サイトが公開)される。

\hypertarget{ux4ecaux5f8cux306eux904bux7528}{%
\subsection{今後の運用}\label{ux4ecaux5f8cux306eux904bux7528}}

一人が文書を変更して適応した後、細かいミスなどで再編集するのは手間。また間違ってファイルを削除してしまった時の危険性も高い。

そのため、後々はプルリクエストを出してもう一人が確認する形式にしようと思う。

プルリクエストは手軽に人の編集を確認・訂正できるgithubの機能である。

\bookmarksetup{startatroot}

\hypertarget{ux5e8fux7ae0ux306eux307eux3068ux3081}{%
\chapter{序章のまとめ}\label{ux5e8fux7ae0ux306eux307eux3068ux3081}}

\hypertarget{ux7b2cuxff11ux7248ux5e8fux6587}{%
\subsection{第1版序文}\label{ux7b2cuxff11ux7248ux5e8fux6587}}

\textbf{強化学習}は1979年ごろから存在が知られてきた。

しかし、筆者は\textbf{環境からどのようにして学習するか}、ということにさほど注目されていないことに気がついた。

\bookmarksetup{startatroot}

\hypertarget{ux7b2cuxff11ux7ae0ux5f37ux5316ux5b66ux7fd2ux3068ux306f}{%
\chapter{第1章:強化学習とは}\label{ux7b2cuxff11ux7ae0ux5f37ux5316ux5b66ux7fd2ux3068ux306f}}

\bookmarksetup{startatroot}

\hypertarget{ux7b2cuxff12ux7ae0}{%
\chapter{第2章}\label{ux7b2cuxff12ux7ae0}}

\bookmarksetup{startatroot}

\hypertarget{ux7b2cuxff13ux7ae0}{%
\chapter{第3章}\label{ux7b2cuxff13ux7ae0}}



\end{document}
